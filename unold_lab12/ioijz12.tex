\documentclass[11pt, a4paper]{article}
\usepackage{polski}
\usepackage[utf8]{inputenc}
\usepackage[T1]{fontenc}

\usepackage{graphicx}

\usepackage{amsmath} 

\usepackage{listings}

\usepackage{color}

\usepackage{marvosym}

\usepackage{geometry}
\usepackage{float}
\usepackage{booktabs}
\usepackage{multirow}
\usepackage{titlesec}
\usepackage{hyperref}

\geometry{margin=1.2in}
\usepackage[final]{pdfpages}

\newcommand{\fbi}{\leavevmode{\parindent=1em\indent}}

\definecolor{dkgreen}{rgb}{0,0.6,0}
\definecolor{gray}{rgb}{0.5,0.5,0.5}
\definecolor{mauve}{rgb}{0.58,0,0.82}

\lstset{
	frame=tblr,
	language=R,
	aboveskip=3mm,
	belowskip=3mm,
	showstringspaces=false,
	columns=flexible,
	basicstyle={\small\ttfamily},
	numbers=left,
	numberstyle=\tiny\color{gray},
	keywordstyle=\color{blue},
	commentstyle=\color{dkgreen},
	stringstyle=\color{mauve},
	breaklines=true,
	mathescape=false,
	breakatwhitespace=true,
	tabsize=3
}

\renewcommand\lstlistingname{Listing}

\titleclass{\subsubsubsection}{straight}[\subsection]
\newcounter{subsubsubsection}[subsubsection]
\renewcommand\thesubsubsubsection{\thesubsubsection.\arabic{subsubsubsection}}
\renewcommand\theparagraph{\thesubsubsubsection.\arabic{paragraph}}

\titleformat{\subsubsubsection}
  {\normalfont\normalsize\bfseries}{\thesubsubsubsection}{1em}{}
\titlespacing*{\subsubsubsection}
{0pt}{3.25ex plus 1ex minus .2ex}{1.5ex plus .2ex}

\makeatletter
\renewcommand\paragraph{\@startsection{paragraph}{5}{\z@}%
  {3.25ex \@plus1ex \@minus.2ex}%
  {-0em}%
  {\normalfont\normalsize\bfseries}}
\renewcommand\subparagraph{\@startsection{subparagraph}{6}{\parindent}%
  {3.25ex \@plus1ex \@minus .2ex}%
  {-1em}%
  {\normalfont\normalsize\bfseries}}
\def\toclevel@subsubsubsection{4}
\def\toclevel@paragraph{5}
\def\toclevel@paragraph{6}
\def\l@subsubsubsection{\@dottedtocline{4}{7em}{4em}}
\def\l@paragraph{\@dottedtocline{5}{10em}{5em}}
\def\l@subparagraph{\@dottedtocline{6}{14em}{6em}}
\makeatother

\setcounter{secnumdepth}{4}
\setcounter{tocdepth}{4}

\hypersetup{pageanchor=false}

\setlength\parindent{3pt}

\renewcommand{\labelenumi}{\alph{enumi}.} 

\date{\today}

\begin{document}

\input{./ioijz12title.tex}

\tableofcontents

\newpage
\section{Wprowadzenie}
\paragraph{}
Algorytmy genetyczne to

\fbi
W ramach laboratorium należało przeprowadzić testy algorytmu genetycznego dla różnych parametrów. Jako benchmark oceny należało użyć pakietu ,,getGlobalOpts'' oraz języka R.

\fbi
Pomiary wykonywano na 2 różnych jednostkach roboczych. Ich parametry nie są istotne z punktu widzenia analizy i możliwości porównania rezultatów.

\section{Implementacja}
\paragraph{}
Poniżej (listing ~\ref{lst:skryptGlowny}) zamieszczono kod napisany w języku R przygotowany w celu umożliwienia przeprowadzenia pomiarów.

\lstinputlisting[label=lst:skryptGlowny,caption=Skrypt w języku R wykorzystany do badań,firstline=1,lastline=300]{./assets/skrypt_lab12.R}


\subsection{Parametryzacja skryptu}
Parametryzacji podlega jedynie algorytm genetyczny.
Wybór funkcji do optymalizacji odbywa się przez podanie jej nazwy.
Pozostałe dane są odczytywane z pakietu ,,globalOptTests''.


\section{Przebieg badań}
\paragraph{}
Do badań zostały wybrane funkcje o różnych wymiarach zaczynając na 2 kończąc na 20. Poniżej wymieniono te funkcje wraz z ilością wymiarów podaną w nawiasie.

\begin{itemize}
	\item Branin (2)
	\item Gulf (3)
	\item CosMix4 (4)
	\item EMichalewicz (5)
	\item Hartman6 (6)
	\item PriceTransistor (9)
	\item Schwefel (10)
	\item Zeldasine20 (20)
\end{itemize}

\fbi
Każdy pomiar przeprowadzano 10-krotnie wyniki uśredniając. Domyślne parametry wynosiły kolejno:

\begin{itemize}
	\item rozmiar populacji - 50
	\item liczba iteracji - 100
	\item p. mutacji - 0,1
	\item p. krzyżowania - 0,8
\end{itemize}


\subsection{Branin (2 parametry)}
\paragraph{}
Branin jest funkcją z dwoma parametrami. Na ilustracji (rys.~\ref{fig:branin1}) przedstawiono jej wykres.

\begin{figure}[H]
	\begin{center}
		\includegraphics[width=0.9\textwidth]{./assets/branin1.png} % 
		\caption{Wykres funkcji Branin (d=2)}
		\label{fig:branin1}
	\end{center}
\end{figure}

\fbi
Na kolejnych stronach zamieszczono wyniki pomiarów dla różnych wartości parametrów algorytmu genetycznego.

\begin{figure}[H]
	\begin{center}
		\includegraphics[width=0.9\textwidth]{./assets/branin2.png} % 
		\caption{Wartość znalezionego optimum w zależności od prawdopodobieństwa mutacji}
		\label{fig:branin2}
	\end{center}
\end{figure}

\begin{figure}[H]
	\begin{center}
		\includegraphics[width=0.9\textwidth]{./assets/branin3.png} % 
		\caption{Wartość znalezionego optimum w zależności od prawdopodobieństwa krzyżowanie}
		\label{fig:branin3}
	\end{center}
\end{figure}

\begin{figure}[H]
	\begin{center}
		\includegraphics[width=0.9\textwidth]{./assets/branin4.png} % 
		\caption{Wartość znalezionego optimum w zależności od przyjętego elityzmu}
		\label{fig:branin4}
	\end{center}
\end{figure}

\begin{figure}[H]
	\begin{center}
		\includegraphics[width=0.9\textwidth]{./assets/branin5.png} % 
		\caption{Wartość znalezionego optimum w zależności od rozmiarów populacji}
		\label{fig:branin5}
	\end{center}
\end{figure}

\begin{figure}[H]
	\begin{center}
		\includegraphics[width=0.9\textwidth]{./assets/branin6.png} % 
		\caption{Wartość znalezionego optimum w zależności od ilości iteracji}
		\label{fig:branin6}
	\end{center}
\end{figure}

\begin{figure}[H]
	\begin{center}
		\includegraphics[width=0.9\textwidth]{./assets/branin7.png} % 
		\caption{Poglądowa lokalizacja najlepszego znalezionego optimum}
		\label{fig:branin7}
	\end{center}
\end{figure}

\subsection{Gulf (3 parametry)}
\paragraph{}
Gulf jest funkcją przyjmującą trzy parametry. Na ilustracji (rys.~\ref{fig:gulf1}) przedstawiono jej wykres dla pierwszych dwóch wymiarów.

\begin{figure}[H]
	\begin{center}
		\includegraphics[width=0.9\textwidth]{./assets/gulf1.png} % 
		\caption{Wykres funkcji Gulf (d=3)}
		\label{fig:gulf1}
	\end{center}
\end{figure}

\fbi
Na kolejnych stronach zamieszczono wyniki pomiarów dla różnych wartości parametrów algorytmu genetycznego.

\begin{figure}[H]
	\begin{center}
		\includegraphics[width=0.9\textwidth]{./assets/gulf2.png} % 
		\caption{Wartość znalezionego optimum w zależności od prawdopodobieństwa mutacji}
		\label{fig:gulf2}
	\end{center}
\end{figure}

\begin{figure}[H]
	\begin{center}
		\includegraphics[width=0.9\textwidth]{./assets/gulf3.png} % 
		\caption{Wartość znalezionego optimum w zależności od prawdopodobieństwa krzyżowanie}
		\label{fig:gulf3}
	\end{center}
\end{figure}

\begin{figure}[H]
	\begin{center}
		\includegraphics[width=0.9\textwidth]{./assets/gulf4.png} % 
		\caption{Wartość znalezionego optimum w zależności od przyjętego elityzmu}
		\label{fig:gulf4}
	\end{center}
\end{figure}

\begin{figure}[H]
	\begin{center}
		\includegraphics[width=0.9\textwidth]{./assets/gulf5.png} % 
		\caption{Wartość znalezionego optimum w zależności od rozmiarów populacji}
		\label{fig:gulf5}
	\end{center}
\end{figure}

\begin{figure}[H]
	\begin{center}
		\includegraphics[width=0.9\textwidth]{./assets/gulf6.png} % 
		\caption{Wartość znalezionego optimum w zależności od ilości iteracji}
		\label{fig:gulf6}
	\end{center}
\end{figure}

\fbi
Jak możemy zauważyć na ilustracji poniżej (rys.~\ref{fig:gulf7}) przedstawiona lokalizacja optimum nie jest poprawna, gdyż optymalizacji poddano wersję z 3 parametrami. Ogólnie rzecz biorąc gdyby 3 wymiar przedstawić w postaci gradientu kolorystycznego wtedy byłaby to poprawna lokalizacja niemniej trudna dla intuicyjnego sprawdzenia.

\begin{figure}[H]
	\begin{center}
		\includegraphics[width=0.9\textwidth]{./assets/gulf7.png} % 
		\caption{Poglądowa lokalizacja najlepszego znalezionego optimum}
		\label{fig:gulf7}
	\end{center}
\end{figure}


\subsection{CosMix4 (4 parametry)}
\paragraph{}
Test

\subsection{EMichalewicz (5 parametrów)}
\paragraph{}
Test

\subsection{Hartman6 (6 parametrów)}
\paragraph{}
Test

\subsection{PriceTransistor (9 parametrów)}
\paragraph{}
Test

\subsection{Schwefel (10 parametrów)}
\paragraph{}
Test

\subsection{Zeldasine20 (20 parametrów)}
\paragraph{}
Test


\section{Podsumowanie}
\paragraph{}
Test

\fbi
Akapit



\newpage
\begin{thebibliography}{40}

\bibitem{test1}
Artur Suchwałko “Wprowadzenie do R dla programistów innych języków” https://cran.r-project.org/doc/contrib/R-dla-programistow-innych-jezykow.pdf

\end{thebibliography}

\end{document}