\documentclass[11pt, a4paper]{article}
\usepackage{polski}
\usepackage[utf8]{inputenc}
\usepackage[T1]{fontenc}
\usepackage[export]{adjustbox}
\usepackage{graphicx}
\usepackage{amsmath} 
\usepackage{listings}
\usepackage{color}
\usepackage{marvosym}
\usepackage{geometry}
\usepackage{float}
\usepackage{booktabs}
\usepackage{multirow}
\usepackage{titlesec}
\usepackage{hyperref}
\usepackage{tabularx}

\geometry{margin=1.2in}
\usepackage[final]{pdfpages}

\newcommand{\fbi}{\leavevmode{\parindent=1em\indent}}

\definecolor{dkgreen}{rgb}{0,0.6,0}
\definecolor{gray}{rgb}{0.5,0.5,0.5}
\definecolor{mauve}{rgb}{0.58,0,0.82}

\lstset{
	frame=tblr,
	language=R,
	aboveskip=3mm,
	belowskip=3mm,
	showstringspaces=false,
	columns=flexible,
	basicstyle={\small\ttfamily},
	numbers=left,
	numberstyle=\tiny\color{gray},
	keywordstyle=\color{blue},
	commentstyle=\color{dkgreen},
	stringstyle=\color{mauve},
	breaklines=true,
	mathescape=false,
	breakatwhitespace=true,
	tabsize=3
}

\renewcommand\lstlistingname{Listing}

\titleclass{\subsubsubsection}{straight}[\subsection]
\newcounter{subsubsubsection}[subsubsection]
\renewcommand\thesubsubsubsection{\thesubsubsection.\arabic{subsubsubsection}}
\renewcommand\theparagraph{\thesubsubsubsection.\arabic{paragraph}}

\titleformat{\subsubsubsection}
  {\normalfont\normalsize\bfseries}{\thesubsubsubsection}{1em}{}
\titlespacing*{\subsubsubsection}
{0pt}{3.25ex plus 1ex minus .2ex}{1.5ex plus .2ex}

\makeatletter
\renewcommand\paragraph{\@startsection{paragraph}{5}{\z@}
  {3.25ex \@plus1ex \@minus.2ex}
  {-0em}
  {\normalfont\normalsize\bfseries}}
\renewcommand\subparagraph{\@startsection{subparagraph}{6}{\parindent}
  {3.25ex \@plus1ex \@minus .2ex}
  {-1em}
  {\normalfont\normalsize\bfseries}}
\def\toclevel@subsubsubsection{4}
\def\toclevel@paragraph{5}
\def\toclevel@paragraph{6}
\def\l@subsubsubsection{\@dottedtocline{4}{7em}{4em}}
\def\l@paragraph{\@dottedtocline{5}{10em}{5em}}
\def\l@subparagraph{\@dottedtocline{6}{14em}{6em}}
\makeatother

\setcounter{secnumdepth}{4}
\setcounter{tocdepth}{4}

\hypersetup{pageanchor=false}

\setlength\parindent{3pt}

\renewcommand{\labelenumi}{\alph{enumi}.} 

\date{\today}

\begin{document}

\input{./ioijz12title.tex}

\tableofcontents

\newpage
\section{Wprowadzenie}
\paragraph{}
Algorytm genetyczny to algorytm heurystyczny...[]

\fbi
W ramach laboratorium należało przeprowadzić testy algorytmu genetycznego dla różnych parametrów. Jako benchmark oceny należało użyć pakietu ,,getGlobalOpts'' oraz języka R.

\fbi
Pomiary wykonywano na 2 różnych jednostkach roboczych. Ich parametry nie są istotne z punktu widzenia analizy i możliwości porównania rezultatów.

\section{Implementacja}
\paragraph{}
Poniżej (listing ~\ref{lst:skryptGlowny}) zamieszczono kod napisany w języku R przygotowany w celu umożliwienia przeprowadzenia pomiarów.

\lstinputlisting[label=lst:skryptGlowny,caption=Skrypt w języku R wykorzystany do badań,firstline=1,lastline=300]{./assets/skrypt_lab12.R}

\subsection{Parametryzacja skryptu}
Parametryzacji podlega jedynie algorytm genetyczny.
Wybór funkcji do optymalizacji odbywa się przez podanie jej nazwy.
Pozostałe dane są odczytywane z pakietu ,,globalOptTests''.
[todo: dopisać o pętli przechodzącej po wszystkich funkcjach oraz po wszystkich parametrach domyślnych]

\newpage
\section{Przebieg badań}
\paragraph{}
Do badań zostały wybrane funkcje o różnych wymiarach zaczynając na 2 kończąc na 20. Poniżej wymieniono te funkcje wraz z ilością wymiarów podaną w nawiasie.

\begin{itemize}
	\item Branin (2)
	\item Gulf (3)
	\item CosMix4 (4)
	\item EMichalewicz (5)
	\item Hartman6 (6)
	\item PriceTransistor (9)
	\item Schwefel (10)
	\item Zeldasine20 (20)
\end{itemize}

\fbi
Każdy pomiar przeprowadzano 20-krotnie wyniki uśredniając co oznacza, że wartości widoczne na wykresach dla każdej serii z osobna są uśrednione po osobnych 20 przebiegach. Domyślne parametry każdej z serii przedstawiono poniżej (tabela~\ref{tab:parametry}). Zmianie ulegają wartości  prawdopodobieństwa mutacji i krzyżowania by zbadać znaczenie ich obecności podczas optymalizacji.

\begin{table}[htbp]
	\centering
	\caption{Parametry domyślne poszczególnych serii pomiarowych}
	\label{tab:parametry}
	\begin{tabularx}{\textwidth}{|X|c|c|c|c|}
		\hline
		- & Seria 1 & Seria 2 & Seria 3 & Seria 4\\ 
		\hline
		Rozmiar populacji & 50 & 50 & 50 & 50 \\ 
		\hline 
		Rozmiar iteracji & 100 & 100 & 100 & 100 \\ 
		\hline 
		Prawdopodobieństwo mutacji & 0 & 0 & 0.1 & 0.1 \\ 
		\hline 
		Prawdopodobieństwo krzyżowania & 0 & 0.8 & 0 & 0.8 \\ 
		\hline 
	\end{tabularx} 
\end{table}

\fbi
Zielone linie na wykresach oznaczają optima zawarte w pakiecie ,,globalOptTests'' dla danej funkcji przy domyślnych ograniczeniach (tych samych dla których wykonywana jest optymalizacja podczas niniejszych pomiarów).

\fbi
Dla funkcji o ilości parametrów większej niż 2 pominięto ilustracje graficzne znalezionych optimów gdyż optymalizacji podlegają wszystkie wymiary a przedstawienie dwóch pierwszych nie niesie ze sobą przydatnej informacji.

\newpage
\subsection{Branin (2 parametry)}
\paragraph{}
Branin jest funkcją z dwoma parametrami. Na ilustracji (rys.~\ref{fig:branin1}) przedstawiono jej wykres a poniżej jej wzór (\ref{eq:branin}).

\begin{equation}\label{eq:branin}
	f(\boldsymbol{x}) = a(x_2 - bx_1^2 + cx_1 - r)^2 + s(1 - t)\cos(x_1) + s
\end{equation}

, gdzie $ x_1 \in [-5, 10] $ oraz $ x_2 \in [0, 15] $.

\begin{figure}[H]
	\centering
	\includegraphics[width=0.95\textwidth]{./assets/Branin1.png}
	\caption{Wykres funkcji Branin}
	\label{fig:branin1}
\end{figure}

\fbi
Powyższy wykres (rys.~\ref{fig:branin1}) pokazuje trójwymiarowy obraz funkcji Branin. Wynika z niego, że funkcja ta ma stosunkowo duży obszar w którym może znajdować się minimum oraz dwie strefy w których wartości są dużo większe.

\fbi
Na kolejnych stronach zamieszczono wyniki pomiarów dla różnych wartości parametrów algorytmu genetycznego. Kolejno dokonano pomiarów dla różnych wartości: prawdopodobieństwa mutacji i krzyżowania, wielkości populacji, ilości iteracji oraz elityzmu. Wszystkie pomiary wykonano dla 4 różnych ustawień domyślnych parametrów.

\newpage
\begin{figure}[H]
	\centering
	\includegraphics[width=0.85\textwidth]{./assets/Branin2.png}
	\caption{Wartość znalezionego minimum funkcji Branin w zależności od prawdopodobieństwa mutacji}
	\label{fig:branin2}
\end{figure}

\begin{figure}[H]
	\centering
	\includegraphics[width=0.85\textwidth]{./assets/Branin3.png}
	\caption{Wartość znalezionego minimum funkcji Branin w zależności od prawdopodobieństwa krzyżowania}
	\label{fig:branin3}
\end{figure}

\fbi
Na wykresie (rys.~\ref{fig:branin2}) można zauważyć niski wpływ ustawienia mutacji na znalezione rozwiązania. Przy wszystkich parametrach domyślnych funkcja znajduje się w pobliżu optymalnej wartości. Miejscowe odchylenia są tu najprawdopodobniej związane z charakterem algorytmu i zbyt małą ilością prób poddanych uśrednieniu. Nie możemy tutaj określić czy przy wyłączonej zarówno mutacji jak i krzyżowaniu wyniki ulegają pogorszeniu, gdyż nie ma w tym obszarze spójności. Podobne wnioski możemy wskazać dla wykresu krzyżowania (rys.~\ref{fig:branin3}).

\begin{figure}[H]
	\centering
	\includegraphics[width=0.85\textwidth]{./assets/Branin4.png}
	\caption{Wartość znalezionego minimum funkcji Branin w zależności od rozmiaru populacji}
	\label{fig:branin4}
\end{figure}

\begin{figure}[H]
	\centering
	\includegraphics[width=0.85\textwidth]{./assets/Branin5.png}
	\caption{Wartość znalezionego minimum funkcji Branin w zależności od ilości iteracji}
	\label{fig:branin5}
\end{figure}

\fbi
Z wykresu (rys.~\ref{fig:branin4}) można odczytać podatność funkcji na zmiany rozmiaru populacji. Wyniki zbliżone do oczekiwanych zostały uzyskane dla wartości wynoszącej 45 jednostek. Widać również, że przy małej populacji znaczenie mutacji i krzyżowania jest większe. Zauważalny jest wzrost jakości rozwiązania wraz ze wzrostem ilości jednostek populacji.

\fbi
Wykres (rys.~\ref{fig:branin5}) wskazuje wyraźną zmianę jakości rozwiązań dla 60 i więcej iteracji. Poniżej tej wartości uzyskiwane wyniki są niestabilne, powyżej osiągają wartość zbliżoną do oczekiwanej szczególnie dla serii 4 (czyli z włączoną mutacją i krzyżowaniem).

\begin{figure}[H]
	\centering
	\includegraphics[width=0.85\textwidth]{./assets/Branin6.png}
	\caption{Wartość znalezionego minimum funkcji Branin w zależności od przyjętego elityzmu}
	\label{fig:branin6}
\end{figure}

\fbi
Z wykonanych pomiarów (rys.~\ref{fig:branin6}) wynika, że dla uzyskania optymalnego rozwiązania należy zastosować wartość elityzmu na poziomie przynajmniej 0.4. Jego ustawienie poniżej tej wartości powoduje obniżenie się jakości rezultatów.

\begin{figure}[H]
	\centering
	\includegraphics[width=0.7\textwidth]{./assets/Branin6elt.png}
	\caption{Poglądowa lokalizacja najlepszego znalezionego minimum funkcji Branin dla pomiarów przy zmianach elityzmu}
	\label{fig:branin6elt}
\end{figure}

\newpage
\subsection{Gulf (3 parametry)}
\paragraph{}
Gulf jest funkcją określoną dla ilości parametrów równej 3. Na ilustracji (rys.~\ref{fig:gulf1}) przedstawiono jej wykres dla pierwszych dwóch.

\begin{equation}\label{eq:gulf}
f(\boldsymbol{x}) = \sum_{i=1}^{99} [\exp(- \frac{(u_i - x_2)^{x_3}}{x_1}) - 0.01i]^2
\end{equation}

, gdzie $ x_1 \in [0.1, 100] $, $ x_2 \in [0, 25.6] $, $ x_3 \in [0, 5] $ oraz $ u_i = 25 + [-50 \ln (0.01i)]^{1/1.5} $.

\begin{figure}[H]
	\begin{center}
		\includegraphics[width=0.95\textwidth]{./assets/Gulf1.png}
		\caption{Wykres funkcji Gulf dla dwóch pierwszych parametrów}
		\label{fig:gulf1}
	\end{center}
\end{figure}

\fbi
Na kolejnych stronach zamieszczono wyniki pomiarów dla różnych wartości parametrów algorytmu genetycznego.

\begin{figure}[H]
	\begin{center}
		\includegraphics[width=0.85\textwidth]{./assets/Gulf2.png}
		\caption{Wartość znalezionego minimum dla funkcji Gulf w zależności od prawdopodobieństwa mutacji}
		\label{fig:gulf2}
	\end{center}
\end{figure}

\fbi
Wartości funkcji Gulf dla zadanego prawdopodobieństwa mutacji są zbliżone do wartości oczekiwanej w przedziale 0.1-0.6. Powyżej tego przedziału mutacja wywiera negatywny wpływ na otrzymywane wyniki.


\begin{figure}[H]
	\begin{center}
		\includegraphics[width=0.85\textwidth]{./assets/Gulf3.png}
		\caption{Wartość znalezionego minimum dla funkcji Gulf w zależności od prawdopodobieństwa krzyżowania}
		\label{fig:gulf3}
	\end{center}
\end{figure}

\fbi
Prawdopodobieństwo krzyżowania ma niski oraz niestabilny wpływ na otrzymane wyniki. Wspólnie (dla wszystkich ustawień domyślnych) najlepsze wyniki uzyskane zostały w przedziale 0.4-0.6. Przyjęcie wartości krzyżowania wykraczających poza wskazany przedział znacząco obniża jakoś uzyskanych wyników.

\begin{figure}[H]
	\begin{center}
		\includegraphics[width=0.85\textwidth]{./assets/Gulf4.png}
		\caption{Wartość znalezionego minimum dla funkcji Gulf w zależności od rozmiarów populacji}
		\label{fig:gulf4}
	\end{center}
\end{figure}

\fbi
Wykres ten wyraznie obrazuje pozytywny wpływ zwiększenia populacji na jakość wyników. Najlepsze wyniki uzyskano dla populacji wynoszącej przynajmniej 50 jednostek.

\fbi
Zauważalne jest również pogorszenie wyników w przedziale 65-80[todo: dlaczego].

\begin{figure}[H]
	\begin{center}
		\includegraphics[width=0.85\textwidth]{./assets/Gulf5.png} 
		\caption{Wartość znalezionego minimum dla funkcji Gulf w zależności od ilości iteracji}
		\label{fig:gulf5}
	\end{center}
\end{figure}

\fbi
Na wykresie można zauważyć znaczące poprawienie się rezultatów, gdy ilość iteracji wynosi przynamniej 80. Poniżej tej wartości uzyskane wyniki są znacząco gorsze od optymalnego rozwiązania. 

\fbi
W przedziale 130-200 [todo: co się dzieje]


\begin{figure}[H]
	\begin{center}
		\includegraphics[width=0.85\textwidth]{./assets/Gulf6.png}
		\caption{Wartość znalezionego minimum dla funkcji Gulf w zależności od przyjętego elityzmu}
		\label{fig:gulf6}
	\end{center}
\end{figure}

\fbi
W przypadku funkcji Gulf elityzm ma znaczący wpływ na otrzymywane wyniki. W celu ich optymalizacji wymagana jest wartość elityzmu na poziomie przynajmniej 0.4.

\fbi
Dla wartości powyżej 0.6 wyniki zaczynają się pogarszać. [todo: dlaczego]


\fbi
[todo: zmienic]
Jak możemy zauważyć na ilustracji poniżej (rys.~\ref{fig:gulf7}) przedstawiona lokalizacja optimum nie jest poprawna, gdyż optymalizacji poddano wersję z 3 parametrami. Ogólnie rzecz biorąc gdyby 3 wymiar przedstawić w postaci gradientu kolorystycznego wtedy byłaby to poprawna lokalizacja niemniej trudna dla intuicyjnego sprawdzenia.

\newpage
\subsection{CosMix4 (4 parametry)}
\paragraph{}
CosMix4 jest funkcją określoną dla ilości parametrów równej 4. Na ilustracji (rys.~\ref{fig:cosmix41}) przedstawiono jej wykres dla pierwszych dwóch.

\begin{equation}\label{eq:cosmix4}
f(\boldsymbol{x}) = -0.1 \sum_{i=1}^{4} \cos (5 \pi x_i) - \sum_{i=1}^{4} x_i^2
\end{equation}

, gdzie $ x_i \in [-2, 1] $ oraz $ i \in {1,2,3,4} $.

\begin{figure}[H]
	\begin{center}
		\includegraphics[width=0.95\textwidth]{./assets/CosMix41.png}
		\caption{Wykres funkcji CosMix4 dla dwóch pierwszych parametrów}
		\label{fig:cosmix41}
	\end{center}
\end{figure}

\fbi
Jak widać funkcja ma dużo lokalnych optimów.

\begin{figure}[H]
	\begin{center}
		\includegraphics[width=0.85\textwidth]{./assets/CosMix42.png}
		\caption{Wartość znalezionego minimum dla funkcji CosMix4 w zależności od prawdopodobieństwa mutacji}
		\label{fig:cosmix42}
	\end{center}
\end{figure}

\fbi
Ogólnie rzecz biorąc, niezależnie od pozostałych parametrów, jedyną niekorzystną sytuacją jest tu jednoczesne wyłączenie mutacji i krzyżowania co możemy zaobserwować na przykładzie serii 1-szej.

\begin{figure}[H]
	\begin{center}
		\includegraphics[width=0.85\textwidth]{./assets/CosMix43.png}
		\caption{Wartość znalezionego minimum dla funkcji CosMix4 w zależności od prawdopodobieństwa krzyżowania}
		\label{fig:cosmix43}
	\end{center}
\end{figure}

\fbi
Podobnie jak na poprzednim wykresie (rys. \ref{fig:cosmix42}) tak i na powyższym (rys. \ref{fig:cosmix43}) ujawnia się niekorzystny wpływ wyłączenia mutacji i krzyżowania.


\begin{figure}[H]
	\begin{center}
		\includegraphics[width=0.85\textwidth]{./assets/CosMix44.png}
		\caption{Wartość znalezionego minimum dla funkcji CosMix4 w zależności od rozmiarów populacji}
		\label{fig:cosmix44}
	\end{center}
\end{figure}

\begin{figure}[H]
	\begin{center}
		\includegraphics[width=0.85\textwidth]{./assets/CosMix45.png}
		\caption{Wartość znalezionego minimum dla funkcji CosMix4 w zależności od ilości iteracji}
		\label{fig:cosmix45}
	\end{center}
\end{figure}

\fbi
Na dwóch poprzedzających wykresach możemy zaobserwować, że domyślne wartości w postaci wielkości populacji w liczbie 50 i ilości iteracji równej 100 są wzajemnie optymalne.

\begin{figure}[H]
	\begin{center}
		\includegraphics[width=0.85\textwidth]{./assets/CosMix46.png}
		\caption{Wartość znalezionego minimum dla funkcji CosMix4 w zależności od przyjętego elityzmu}
		\label{fig:cosmix46}
	\end{center}
\end{figure}

\newpage
\subsection{EMichalewicz (5 parametrów)}
\paragraph{}
Poniżej zamieszczono wzór rozpatrywanej funkcji.

\begin{equation}\label{eq:emichalewicz}
f(\boldsymbol{x}) = - \sum_{i=1}^{d} \sin(x_i) \sin^{2m} (\frac{i x_i^2}{\pi})
\end{equation}

\begin{figure}[H]
	\begin{center}
		\includegraphics[width=0.95\textwidth]{./assets/EMichalewicz1.png}
		\caption{Wykres funkcji EMichalewicz (d=5)}
		\label{fig:emichalewicz1}
	\end{center}
\end{figure}

\begin{figure}[H]
	\begin{center}
		\includegraphics[width=0.85\textwidth]{./assets/EMichalewicz2.png}
		\caption{Wartość znalezionego minimum dla funkcji EMichalewicz w zależności od prawdopodobieństwa mutacji}
		\label{fig:emichalewicz2}
	\end{center}
\end{figure}

\begin{figure}[H]
	\begin{center}
		\includegraphics[width=0.85\textwidth]{./assets/EMichalewicz3.png}
		\caption{Wartość znalezionego minimum dla funkcji EMichalewicz w zależności od prawdopodobieństwa krzyżowania}
		\label{fig:emichalewicz3}
	\end{center}
\end{figure}

\begin{figure}[H]
	\begin{center}
		\includegraphics[width=0.85\textwidth]{./assets/EMichalewicz4.png}
		\caption{Wartość znalezionego minimum dla funkcji EMichalewicz w zależności od rozmiarów populacji}
		\label{fig:emichalewicz4}
	\end{center}
\end{figure}

\begin{figure}[H]
	\begin{center}
		\includegraphics[width=0.85\textwidth]{./assets/EMichalewicz5.png}
		\caption{Wartość znalezionego minimum dla funkcji EMichalewicz w zależności od ilości iteracji}
		\label{fig:emichalewicz5}
	\end{center}
\end{figure}

\begin{figure}[H]
	\begin{center}
		\includegraphics[width=0.85\textwidth]{./assets/EMichalewicz6.png}
		\caption{Wartość znalezionego minimum dla funkcji EMichalewicz w zależności od przyjętego elityzmu}
		\label{fig:emichalewicz6}
	\end{center}
\end{figure}

\newpage
\subsection{Hartman6 (6 parametrów)}
\paragraph{}
Hartman6 jest funkcją określoną dla ilości parametrów równej 6. Na ilustracji (rys.~\ref{fig:hartman6}) przedstawiono jej wykres dla pierwszych dwóch.

\begin{equation}\label{eq:hartman6}
f(\boldsymbol{x}) = - \sum_{i=1}^{4} c_i \exp[- \sum_{j=1}^{6} a_{ij}(x_j - p_{ij})^2]
\end{equation}

, gdzie $ x_i \in [0, 1] $, $ i \in \{1, ..., 6\} $.

\begin{figure}[H]
	\begin{center}
		\includegraphics[width=0.95\textwidth]{./assets/Hartman61.png}
		\caption{Wykres funkcji Hartman6 (d=6)}
		\label{fig:hartman61}
	\end{center}
\end{figure}

\begin{figure}[H]
	\begin{center}
		\includegraphics[width=0.85\textwidth]{./assets/Hartman62.png}
		\caption{Wartość znalezionego minimum dla funkcji Hartman6 w zależności od prawdopodobieństwa mutacji}
		\label{fig:hartman62}
	\end{center}
\end{figure}

\begin{figure}[H]
	\begin{center}
		\includegraphics[width=0.85\textwidth]{./assets/Hartman63.png}
		\caption{Wartość znalezionego minimum dla funkcji Hartman6 w zależności od prawdopodobieństwa krzyżowania}
		\label{fig:hartman63}
	\end{center}
\end{figure}

\begin{figure}[H]
	\begin{center}
		\includegraphics[width=0.85\textwidth]{./assets/Hartman64.png}
		\caption{Wartość znalezionego minimum dla funkcji Hartman6 w zależności od rozmiarów populacji}
		\label{fig:hartman64}
	\end{center}
\end{figure}

\begin{figure}[H]
	\begin{center}
		\includegraphics[width=0.85\textwidth]{./assets/Hartman65.png}
		\caption{Wartość znalezionego minimum dla funkcji Hartman6 w zależności od ilości iteracji}
		\label{fig:hartman65}
	\end{center}
\end{figure}

\begin{figure}[H]
	\begin{center}
		\includegraphics[width=0.85\textwidth]{./assets/Hartman66.png}
		\caption{Wartość znalezionego minimum dla funkcji Hartman6 w zależności od przyjętego elityzmu}
		\label{fig:hartman66}
	\end{center}
\end{figure}

\newpage
\subsection{PriceTransistor (9 parametrów)}
\paragraph{}
PriceTransistor jest funkcją określoną dla ilości parametrów równej 9. Na ilustracji (rys.~\ref{fig:pricetransistor1}) przedstawiono jej wykres dla pierwszych dwóch.

\begin{equation}\label{eq:pricetransistor}
f(\boldsymbol{x}) = \gamma^2 + \sum_{k=1}^{4} ( \alpha_k^2 + \beta_k^2 )
\end{equation}

, gdzie $ \alpha_k = (1-x_1 x_2)x_3 {\exp[ x^5 (g_{1k} - g_{3k})]}$.


\begin{figure}[H]
	\begin{center}
		\includegraphics[width=0.95\textwidth]{./assets/PriceTransistor1.png}
		\caption{Wykres funkcji PriceTransistor (d=9)}
		\label{fig:pricetransistor1}
	\end{center}
\end{figure}

\begin{figure}[H]
	\begin{center}
		\includegraphics[width=0.85\textwidth]{./assets/PriceTransistor2.png}
		\caption{Wartość znalezionego minimum dla funkcji PriceTransistor w zależności od prawdopodobieństwa mutacji}
		\label{fig:pricetransistor2}
	\end{center}
\end{figure}

\begin{figure}[H]
	\begin{center}
		\includegraphics[width=0.85\textwidth]{./assets/PriceTransistor3.png}
		\caption{Wartość znalezionego minimum dla funkcji PriceTransistor w zależności od prawdopodobieństwa krzyżowania}
		\label{fig:pricetransistor3}
	\end{center}
\end{figure}

\fbi
Na podstawie powyższych wykresów mutacji (rys.~\ref{fig:pricetransistor2}) oraz krzyżowania (rys.~\ref{fig:pricetransistor3}) możemy uznać, że: w przypadku mutacji wystąpił nieoczekiwany wynik dla wartości 0,5 który ,,spłaszczył'' resztę wykresu, w przypadku krzyżowania najlepsze wyniki uzyskano dla wartości prawdopodobieństwa powyżej 0,6.

\begin{figure}[H]
	\begin{center}
		\includegraphics[width=0.85\textwidth]{./assets/PriceTransistor4.png}
		\caption{Wartość znalezionego minimum dla funkcji PriceTransistor w zależności od rozmiarów populacji}
		\label{fig:pricetransistor4}
	\end{center}
\end{figure}

\begin{figure}[H]
	\begin{center}
		\includegraphics[width=0.85\textwidth]{./assets/PriceTransistor5.png}
		\caption{Wartość znalezionego minimum dla funkcji PriceTransistor w zależności od ilości iteracji}
		\label{fig:pricetransistor5}
	\end{center}
\end{figure}

\fbi
Na podstawie powyższych wykresów populacji (rys.~\ref{fig:pricetransistor4}) oraz iteracji (rys.~\ref{fig:pricetransistor5}) można zauważyć iż wraz ze wzrostem każdego z parametrów wyniki ulegają poprawie.

\begin{figure}[H]
	\begin{center}
		\includegraphics[width=0.85\textwidth]{./assets/PriceTransistor6.png}
		\caption{Wartość znalezionego minimum dla funkcji PriceTransistor w zależności od przyjętego elityzmu}
		\label{fig:pricetransistor6}
	\end{center}
\end{figure}

\fbi
W przypadku rozpatrywanej funkcji widać (rys.~\ref{fig:pricetransistor6}), że jeden z wynik dla jednej z konfiguracji ,,zaburzył'' skalę wykresu.

\newpage
\subsection{Schwefel (10 parametrów)}
\paragraph{}
Schwefel jest funkcją określoną dla ilości parametrów równej 10. Na ilustracji (rys.~\ref{fig:schwefel1}) przedstawiono jej wykres dla pierwszych dwóch.

\begin{equation}\label{eq:schwefel}
f(\boldsymbol{x}) = 418.9829d - \sum_{i=1}^{d} x_i \sin(\sqrt{|x_i|})
\end{equation}

\begin{figure}[H]
	\begin{center}
		\includegraphics[width=0.95\textwidth]{./assets/Schwefel1.png}
		\caption{Wykres funkcji Schwefel (d=10) dla dwóch pierwszych wymiarów}
		\label{fig:schwefel1}
	\end{center}
\end{figure}

\begin{figure}[H]
	\begin{center}
		\includegraphics[width=0.85\textwidth]{./assets/Schwefel2.png}
		\caption{Wartość znalezionego minimum dla funkcji Schwefel w zależności od prawdopodobieństwa mutacji}
		\label{fig:schwefel2}
	\end{center}
\end{figure}

\begin{figure}[H]
	\begin{center}
		\includegraphics[width=0.85\textwidth]{./assets/Schwefel3.png}
		\caption{Wartość znalezionego minimum dla funkcji Schwefel w zależności od prawdopodobieństwa krzyżowania}
		\label{fig:schwefel3}
	\end{center}
\end{figure}

\fbi
Na podstawie powyższych wykresów mutacji (rys.~\ref{fig:schwefel2}) oraz krzyżowania (rys.~\ref{fig:schwefel3}) możemy uznać, że relatywnie najlepsze wyniki uzyskujemy dla p. mutacji rzędu 0,6 -- 0,7 oraz p. krzyżowania rzędu 0,6 lub 0,9.

\begin{figure}[H]
	\begin{center}
		\includegraphics[width=0.85\textwidth]{./assets/Schwefel4.png}
		\caption{Wartość znalezionego minimum dla funkcji Schwefel w zależności od rozmiarów populacji}
		\label{fig:schwefel4}
	\end{center}
\end{figure}

\begin{figure}[H]
	\begin{center}
		\includegraphics[width=0.85\textwidth]{./assets/Schwefel5.png}
		\caption{Wartość znalezionego minimum dla funkcji Schwefel w zależności od ilości iteracji}
		\label{fig:schwefel5}
	\end{center}
\end{figure}

\fbi
Na podstawie powyższych wykresów populacji (rys.~\ref{fig:schwefel4}) oraz iteracji (rys.~\ref{fig:schwefel5}) można zauważyć iż zachodzą pewne prawidłowości lecz nie są one całkowicie zgodne z intuicją jeżeli o takowej możemy tu mówić. Optymalny rozmiar populacji zdaje się wynosić 75. Natomiast ilość iteracji powyżej 150 chwilowo pogarsza wyniki, jednak jak widać dla większych wartości ponownie się one polepszają. Warto zauważyć, że dla ilości iteracji 150 nie osiągane jest optimum zatem przypuszczalnie wzrost ilości iteracji powinien tu pomóc. Wszystko zależy też od tego ile czasu możemy przeznaczyć na poszukiwania.

\begin{figure}[H]
	\begin{center}
		\includegraphics[width=0.85\textwidth]{./assets/Schwefel6.png}
		\caption{Wartość znalezionego minimum dla funkcji Schwefel w zależności od przyjętego elityzmu}
		\label{fig:schwefel6}
	\end{center}
\end{figure}

\fbi
W przypadku rozpatrywanej funkcji najlepsze rezultaty otrzymano (rys.~\ref{fig:schwefel6}) dla wartości elityzmu rzędu 0,6 -- 0,7. Oznacza to, że gdy trochę więcej niż połowa osobników przechodzi do kolejnego pokolenia uzyskujemy najlepsze wyniki.

\fbi
Analizując otrzymane rezultaty całościowo możemy stwierdzić, że w żadnym przypadku nie udało się otrzymać wartości optymalnej. Jest to związane ze stosunkowo dużą przestrzenią poszukiwań i dużą ilością lokalnych optimów.

\newpage
\subsection{Zeldasine20 (20 parametrów)}
\paragraph{}
Zeldasine20 jest funkcją określoną dla ilości parametrów równej 20. Na ilustracji (rys.~\ref{fig:zeldasine1}) przedstawiono jej wykres dla pierwszych dwóch.

\begin{equation}\label{eq:zeldasine}
f(\boldsymbol{x}) = -A \prod_{j=1}^{D} \sin (x_j - z) - \prod_{j=1}^{D} \sin (B \cdot (x_j - z))
\end{equation}

, gdzie $ x_j \in [0, \pi]$ oraz $j = \{1, ..., 10\}$.

\begin{figure}[H]
	\begin{center}
		\includegraphics[width=0.95\textwidth]{./assets/Zeldasine201.png}
		\caption{Wykres funkcji Zeldasine20 dla dwóch pierwszych parametrów}
		\label{fig:zeldasine1}
	\end{center}
\end{figure}

\fbi
Funkcja ma bardzo pofalowaną powierzchnię i dużo lokalnych optimów. Można intuicyjnie założyć, że jest ciężka do optymalizacji.

\begin{figure}[H]
	\begin{center}
		\includegraphics[width=0.85\textwidth]{./assets/Zeldasine202.png}
		\caption{Wartość znalezionego minimum dla funkcji Zeldasine20 w zależności od prawdopodobieństwa mutacji}
		\label{fig:zeldasine2}
	\end{center}
\end{figure}

\begin{figure}[H]
	\begin{center}
		\includegraphics[width=0.85\textwidth]{./assets/Zeldasine203.png}
		\caption{Wartość znalezionego minimum dla funkcji Zeldasine20 w zależności od prawdopodobieństwa krzyżowania}
		\label{fig:zeldasine3}
	\end{center}
\end{figure}

\fbi
Na podstawie powyższych wykresów mutacji (rys.~\ref{fig:zeldasine2}) oraz krzyżowania (rys.~\ref{fig:zeldasine3}) możemy uznać, że relatywnie najlepsze wyniki uzyskujemy dla p. mutacji rzędu 0,4 oraz p. krzyżowania rzędu 0,6.

\begin{figure}[H]
	\begin{center}
		\includegraphics[width=0.85\textwidth]{./assets/Zeldasine204.png}
		\caption{Wartość znalezionego minimum dla funkcji Zeldasine20 w zależności od rozmiarów populacji}
		\label{fig:zeldasine4}
	\end{center}
\end{figure}

\begin{figure}[H]
	\begin{center}
		\includegraphics[width=0.85\textwidth]{./assets/Zeldasine205.png}
		\caption{Wartość znalezionego minimum dla funkcji Zeldasine20 w zależności od ilości iteracji}
		\label{fig:zeldasine5}
	\end{center}
\end{figure}

\fbi
Na podstawie powyższych wykresów populacji (rys.~\ref{fig:zeldasine4}) oraz iteracji (rys.~\ref{fig:zeldasine5}) możemy uznać, że wzrost rozmiaru populacji i ilości iteracji wpływa pozytywnie na jakość rezultatów. Zwłaszcza zwiększanie ilości iteracji w przypadku funkcji z tak dużą ilością parametrów zdaje się prowadzić w dobrym kierunku.

\begin{figure}[H]
	\begin{center}
		\includegraphics[width=0.85\textwidth]{./assets/Zeldasine206.png}
		\caption{Wartość znalezionego minimum dla funkcji Zeldasine20 w zależności od przyjętego elityzmu}
		\label{fig:zeldasine6}
	\end{center}
\end{figure}

\fbi
W przypadku rozpatrywanej funkcji najlepsze rezultaty otrzymano (rys.~\ref{fig:zeldasine6}) dla wartości elityzmu rzędu 0,9 -- 1,0. Oznacza to, że gdy wszystkie osobniki przechodzą do kolejnego pokolenia uzyskujemy najlepsze wyniki.

\fbi
Analizując otrzymane rezultaty możemy stwierdzić, że w żadnym przypadku nie udało się otrzymać wartości bliskiej szukanemu optimum. Jest to związane z dużą przestrzenią poszukiwań. Musimy pamiętać, że rozpatrujemy tu funkcję o 20 parametrach.

\newpage
\section{Podsumowanie}
\paragraph{}
W trakcie prowadzonych badań przetestowano algorytm genetyczny w zadaniu optymalizacji dla 9 funkcji testowych. Analizie poddano wpływ zmiany każdego z parametrów dla 4 różnych konfiguracji pozostałych wartości domyślnych.

\fbi
Wartość prawdopodobieństwa mutacji i krzyżowania zdaje się odgrywać drugorzędną rolę. Istotne jednak by chociaż jedna z nich była włączona z prawdopodobieństwem większym niż 0.

\fbi
Najlepszym ustawieniem dla elityzmu jest prawdopodobieństwo rzędu 0,5.

\fbi
Z pewnością należałoby zwiększyć ilość prób poddawanych uśrednianiu gdyż dla przyjętych 20 wyniki ciągle są niestabilne. Warto by również rozważyć pomijanie kilku najlepszych i najgorszych wyników przed uśrednianiem.

\fbi
Co ciekawe wyniki są widocznie gorsze przy konfiguracji w której krzyżowanie jest wyłączone a p. mutacji wynosi 0,5. Taka prawidłowość objawia się dla wszystkich badanych funkcji.

\newpage
\begin{thebibliography}{40}

\bibitem{test1}
Artur Suchwałko ,,Wprowadzenie do R dla programistów innych języków'' https://cran.r-project.org/doc/contrib/R-dla-programistow-innych-jezykow.pdf

\bibitem{test2}
Luca Scrucca ,,A quick tour of GA''
https://cran.r-project.org/web/packages/GA/vignettes/GA.html

\bibitem{test3}
Surjanovic, S. \& Bingham, D. (2013). Virtual Library of Simulation Experiments: Test Functions and Datasets. Retrieved April 3, 2017, from http://www.sfu.ca/~ssurjano.

\end{thebibliography}

\end{document}