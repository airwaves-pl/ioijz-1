\documentclass[11pt, a4paper]{article}
\usepackage{polski}
\usepackage[utf8]{inputenc}
\usepackage[T1]{fontenc}
\usepackage[export]{adjustbox}
\usepackage{graphicx}
\usepackage{amsmath} 
\usepackage{listings}
\usepackage{color}
\usepackage{marvosym}
\usepackage{geometry}
\usepackage{float}
\usepackage{booktabs}
\usepackage{multirow}
\usepackage{titlesec}
\usepackage{hyperref}
\usepackage{tabularx}

\geometry{margin=1.2in}
\usepackage[final]{pdfpages}

\newcommand{\fbi}{\leavevmode{\parindent=1em\indent}}

\definecolor{dkgreen}{rgb}{0,0.6,0}
\definecolor{gray}{rgb}{0.5,0.5,0.5}
\definecolor{mauve}{rgb}{0.58,0,0.82}

\lstset{
	frame=tblr,
	language=R,
	aboveskip=3mm,
	belowskip=3mm,
	showstringspaces=false,
	columns=flexible,
	basicstyle={\small\ttfamily},
	numbers=left,
	numberstyle=\tiny\color{gray},
	keywordstyle=\color{blue},
	commentstyle=\color{dkgreen},
	stringstyle=\color{mauve},
	breaklines=true,
	mathescape=false,
	breakatwhitespace=true,
	tabsize=3
}

\renewcommand\lstlistingname{Listing}

\titleclass{\subsubsubsection}{straight}[\subsection]
\newcounter{subsubsubsection}[subsubsection]
\renewcommand\thesubsubsubsection{\thesubsubsection.\arabic{subsubsubsection}}
\renewcommand\theparagraph{\thesubsubsubsection.\arabic{paragraph}}

\titleformat{\subsubsubsection}
  {\normalfont\normalsize\bfseries}{\thesubsubsubsection}{1em}{}
\titlespacing*{\subsubsubsection}
{0pt}{3.25ex plus 1ex minus .2ex}{1.5ex plus .2ex}

\makeatletter
\renewcommand\paragraph{\@startsection{paragraph}{5}{\z@}%
  {3.25ex \@plus1ex \@minus.2ex}%
  {-0em}%
  {\normalfont\normalsize\bfseries}}
\renewcommand\subparagraph{\@startsection{subparagraph}{6}{\parindent}%
  {3.25ex \@plus1ex \@minus .2ex}%
  {-1em}%
  {\normalfont\normalsize\bfseries}}
\def\toclevel@subsubsubsection{4}
\def\toclevel@paragraph{5}
\def\toclevel@paragraph{6}
\def\l@subsubsubsection{\@dottedtocline{4}{7em}{4em}}
\def\l@paragraph{\@dottedtocline{5}{10em}{5em}}
\def\l@subparagraph{\@dottedtocline{6}{14em}{6em}}
\makeatother

\setcounter{secnumdepth}{4}
\setcounter{tocdepth}{4}

\hypersetup{pageanchor=false}

\setlength\parindent{3pt}

\renewcommand{\labelenumi}{\alph{enumi}.} 

\date{\today}

\begin{document}

\input{./ioijz12title.tex}

\tableofcontents

\newpage
\section{Wprowadzenie}
\paragraph{}
Algorytmy genetyczne to

\fbi
W ramach laboratorium należało przeprowadzić testy algorytmu genetycznego dla różnych parametrów. Jako benchmark oceny należało użyć pakietu ,,getGlobalOpts'' oraz języka R.

\fbi
Pomiary wykonywano na 2 różnych jednostkach roboczych. Ich parametry nie są istotne z punktu widzenia analizy i możliwości porównania rezultatów.

\section{Implementacja}
\paragraph{}
Poniżej (listing ~\ref{lst:skryptGlowny}) zamieszczono kod napisany w języku R przygotowany w celu umożliwienia przeprowadzenia pomiarów.

\lstinputlisting[label=lst:skryptGlowny,caption=Skrypt w języku R wykorzystany do badań,firstline=1,lastline=300]{./assets/skrypt_lab12.R}

\subsection{Parametryzacja skryptu}
Parametryzacji podlega jedynie algorytm genetyczny.
Wybór funkcji do optymalizacji odbywa się przez podanie jej nazwy.
Pozostałe dane są odczytywane z pakietu ,,globalOptTests''.
[todo: dopisać o pętli przechodzącej po wszystkich funkcjach oraz po wszystkich parametrach domyślnych]

\newpage
\section{Przebieg badań}
\paragraph{}
Do badań zostały wybrane funkcje o różnych wymiarach zaczynając na 2 kończąc na 20. Poniżej wymieniono te funkcje wraz z ilością wymiarów podaną w nawiasie.

\begin{itemize}
	\item Branin (2)
	\item Gulf (3)
	\item CosMix4 (4)
	\item EMichalewicz (5)
	\item Hartman6 (6)
	\item PriceTransistor (9)
	\item Schwefel (10)
	\item Zeldasine20 (20)
\end{itemize}

\fbi
Każdy pomiar przeprowadzano 10-krotnie wyniki uśredniając. Domyślne parametry przedstawiono poniżej (tab. \ref{tab:parametry}).

\begin{table}[htbp]
	\centering
	\caption{Parametry domyślne poszczególnych serii pomiarowych}
	\label{tab:parametry}
	\begin{tabularx}{\textwidth}{|X|c|c|c|}
		\hline
		- & Seria 1 & Seria 2 & Seria 3 \\ 
		\hline
		Rozmiar populacji & 50 & 50 & 25 \\ 
		\hline 
		Rozmiar iteracji & 100 & 100 & 50 \\ 
		\hline 
		Prawdopodobieństwo mutacji & 0 & 0.1 & 0.1 \\ 
		\hline 
		Prawdopodobieństwo krzyżowania & 0 & 0.8 & 0.8 \\ 
		\hline 
	\end{tabularx} 
\end{table}

\fbi
Zielone linie na wykresach oznaczają optima zwracane w pakiecie ,,globalOptTests'' dla danej funkcji przy domyślnych ograniczeniach (tych samych dla których wykonywana jest optymalizacja podczas omawianych badań) .

\subsection{Branin (2 parametry)}
\paragraph{}
Branin jest funkcją z dwoma parametrami. Na ilustracji (rys.~\ref{fig:branin1}) przedstawiono jej wykres. Wzór funkcji zamieszczono poniżej (\ref{eq:branin}).
[todo: opisać dziedzinę]


\begin{equation}\label{eq:branin}
	f(\boldsymbol{x}) = a(x_2 - bx_1^2 + cx_1 - r)^2 + s(1 - t)\cos(x_1) + s
\end{equation}


\begin{figure}[H]
	\centering
	\includegraphics[width=0.9\textwidth]{./assets/Branin1.png}
	\caption{Wykres funkcji Branin (d=2)}
	\label{fig:branin1}
\end{figure}

\fbi
Powyższy wykres pokazuje trójwymiarowy obraz funkcji Branin.
[todo: z którego wynika ...]

\fbi
Na kolejnych stronach zamieszczono wyniki pomiarów dla różnych wartości parametrów algorytmu genetycznego.

\begin{figure}[H]
	\centering
	\includegraphics[width=0.9\textwidth]{./assets/Branin2.png}
	\caption{Wartość znalezionego optimum w zależności od prawdopodobieństwa mutacji}
	\label{fig:branin2}
\end{figure}

\fbi
Na powyższym wykresie można zauważyć niski wpływ mutacji na znalezione rozwiązania. Przy wszystkich parametrach domyślnych funkcja znajduje się w pobliżu optymalnej wartości.

\fbi
Wyjątkiem stanowi tutaj "seria 2" reprezentująca drugi zestaw wartości domyślnych. Przy mutacji wynoszącej 0.7 wynik funkcji znacząco się pogorszył.


\begin{figure}[H]
	\centering
	\includegraphics[width=0.9\textwidth]{./assets/Branin3.png} % 
	\caption{Wartość znalezionego optimum w zależności od prawdopodobieństwa krzyżowania}
	\label{fig:branin3}
\end{figure}

\fbi
Wykres przyjmuje wartości zbliżone do oczekiwanych, gdy prawdopodobieństwo krzyżowania wynosi 0.2 lub więcej.

\fbi
Najlepszy rezultat został otrzymany w przedziale prawdopodobieństwa krzyżowania między 0.4 a 0.6.


\begin{figure}[H]
	\centering
	\includegraphics[width=0.9\textwidth]{./assets/Branin4.png} % 
	\caption{Wartość znalezionego optimum w zależności od rozmiarów populacji}
	\label{fig:branin4}
\end{figure}

\fbi
Z wykresu można odczytać podatność funkcji w zależności od populacji. Wyniki zbliżone do oczekiwanych zostały uzyskane dla populacji wynoszącej 45 jednostek.

\fbi
Zauważalny jest wzrost jakości rozwiązania wraz ze wzrostem ilości jednostek populacji.


\begin{figure}[H]
	\centering
	\includegraphics[width=0.9\textwidth]{./assets/Branin5.png} % 
	\caption{Wartość znalezionego optimum w zależności od ilości iteracji}
	\label{fig:branin5}
\end{figure}

\fbi
Wykres wskazuje wyranźną zmianę jakości rozwiązań dla 75 i więcej iteracji. Poniżej tej wartośći uzyskiwane wyniki są niestabilne, powyżej osiągają wartość zbliżoną do oczekiwanej.

\begin{figure}[H]
	\centering
	\includegraphics[width=0.9\textwidth]{./assets/Branin6.png} % 
	\caption{Wartość znalezionego optimum w zależności od przyjętego elityzmu}
	\label{fig:branin6}
\end{figure}

\fbi
Z wykonanych badań wynika, że do uzyskania optymalnego rozwiązania należy zastosować wartość elityzmu na poziomie przynajmniej 0.4. Jego ustawienie poniżej tej wartości powoduje znaczące obniżenie się jakości rozwiązania.

\fbi
Warto tutaj zauważyć ponowne (jak w przypadku mutacji) obniżenie się jakości wyniku dla przedziału 0.6-0.8.

\newpage
\subsection{Gulf (3 parametry)}
\paragraph{}
Gulf jest funkcją przyjmującą trzy parametry. Na ilustracji (rys.~\ref{fig:gulf1}) przedstawiono jej wykres dla pierwszych dwóch wymiarów.

[todo: dodać wzór, dziedzinę funkcji]


\begin{figure}[H]
	\begin{center}
		\includegraphics[width=0.9\textwidth]{./assets/Gulf1.png} % 
		\caption{Wykres funkcji Gulf (d=3)}
		\label{fig:gulf1}
	\end{center}
\end{figure}

\fbi
Na kolejnych stronach zamieszczono wyniki pomiarów dla różnych wartości parametrów algorytmu genetycznego.

\begin{figure}[H]
	\begin{center}
		\includegraphics[width=0.9\textwidth]{./assets/Gulf2.png} % 
		\caption{Wartość znalezionego optimum w zależności od prawdopodobieństwa mutacji}
		\label{fig:gulf2}
	\end{center}
\end{figure}

\fbi
Wartości funkcji Gulf dla zadanego prawdopodobieństwa mutacji są zbliżone do wartości oczekiwanej w przedziale 0.1-0.6. Powyżej tego przedziału mutacja wywiera negatywny wpływ na otrzymywane wyniki.


\begin{figure}[H]
	\begin{center}
		\includegraphics[width=0.9\textwidth]{./assets/Gulf3.png} % 
		\caption{Wartość znalezionego optimum w zależności od prawdopodobieństwa krzyżowania}
		\label{fig:gulf3}
	\end{center}
\end{figure}

\fbi
Prawdopodobieństwo krzyżowania ma niski oraz niestabilny wpływ na otrzymane wyniki. Wspólnie (dla wszystkich ustawień domyślnych) najlepsze wyniki uzyskane zostały w przedziale 0.4-0.6. Przyjęcie wartości krzyżowania wykraczających poza wskazany przedział znacząco obniża jakoś uzyskanych wyników.


\begin{figure}[H]
	\begin{center}
		\includegraphics[width=0.9\textwidth]{./assets/Gulf4.png} % 
		\caption{Wartość znalezionego optimum w zależności od rozmiarów populacji}
		\label{fig:gulf4}
	\end{center}
\end{figure}

\fbi
Wykres ten wyraznie obrazuje pozytywny wpływ zwiększenia populacji na jakość wyników. Najlepsze wyniki uzyskano dla populacji wynoszącej przynajmniej 50 jednostek.

\fbi
Zauważalne jest również pogorszenie wyników w przedziale 65-80[todo: dlaczego].

\begin{figure}[H]
	\begin{center}
		\includegraphics[width=0.9\textwidth]{./assets/Gulf5.png} % 
		\caption{Wartość znalezionego optimum w zależności od ilości iteracji}
		\label{fig:gulf5}
	\end{center}
\end{figure}

\fbi
Na wykresie można zauważyć znaczące poprawienie się rezultatów, gdy ilość iteracji wynosi przynamniej 80. Poniżej tej wartości uzyskane wyniki są znacząco gorsze od optymalnego rozwiązania. 

\fbi
W przedziale 130-200 [todo: co się dzieje]


\begin{figure}[H]
	\begin{center}
		\includegraphics[width=0.9\textwidth]{./assets/Gulf6.png} % 
		\caption{Wartość znalezionego optimum w zależności od przyjętego elityzmu}
		\label{fig:gulf6}
	\end{center}
\end{figure}

\fbi
W przypadku funkcji Gulf elitzym ma znaczący wpływ na otrzymywane wyniki. W celu ich optymalizacji wymanaja jest wartość elityzmu na poziomie przynajmniej 0.4.

\fbi
Dla wartości powyżej 0.6 wyniki zaczynają się pogarszać. [todo: dlaczego]


\fbi
[todo: zmienic]
Jak możemy zauważyć na ilustracji poniżej (rys.~\ref{fig:gulf7}) przedstawiona lokalizacja optimum nie jest poprawna, gdyż optymalizacji poddano wersję z 3 parametrami. Ogólnie rzecz biorąc gdyby 3 wymiar przedstawić w postaci gradientu kolorystycznego wtedy byłaby to poprawna lokalizacja niemniej trudna dla intuicyjnego sprawdzenia.

\newpage
\subsection{CosMix4 (4 parametry)}
\paragraph{}


\begin{figure}[H]
	\begin{center}
		\includegraphics[width=0.9\textwidth]{./assets/CosMix41.png} % 
		\caption{Wykres funkcji CosMix4 (d=4)}
		\label{fig:cosmix41}
	\end{center}
\end{figure}

\begin{figure}[H]
	\begin{center}
		\includegraphics[width=0.9\textwidth]{./assets/CosMix42.png} % 
		\caption{Wartość znalezionego optimum w zależności od prawdopodobieństwa mutacji}
		\label{fig:cosmix42}
	\end{center}
\end{figure}

\begin{figure}[H]
	\begin{center}
		\includegraphics[width=0.9\textwidth]{./assets/CosMix43.png} % 
		\caption{Wartość znalezionego optimum w zależności od prawdopodobieństwa krzyżowania}
		\label{fig:cosmix43}
	\end{center}
\end{figure}

\begin{figure}[H]
	\begin{center}
		\includegraphics[width=0.9\textwidth]{./assets/CosMix44.png} % 
		\caption{Wartość znalezionego optimum w zależności od rozmiarów populacji}
		\label{fig:cosmix44}
	\end{center}
\end{figure}

\begin{figure}[H]
	\begin{center}
		\includegraphics[width=0.9\textwidth]{./assets/CosMix45.png} % 
		\caption{Wartość znalezionego optimum w zależności od ilości iteracji}
		\label{fig:cosmix45}
	\end{center}
\end{figure}

\begin{figure}[H]
	\begin{center}
		\includegraphics[width=0.9\textwidth]{./assets/CosMix46.png} % 
		\caption{Wartość znalezionego optimum w zależności od przyjętego elityzmu}
		\label{fig:cosmix46}
	\end{center}
\end{figure}

\subsection{EMichalewicz (5 parametrów)}
\paragraph{}
Poniżej zamieszczono wzór rozpatrywanej funkcji.

\begin{equation}\label{eq:emichalewicz}
f(\boldsymbol{x}) = - \sum_{i=1}^{d} \sin(x_i) \sin^{2m} (\frac{i x_i^2}{\pi})
\end{equation}

\newpage
\begin{figure}[H]
	\begin{center}
		\includegraphics[width=0.9\textwidth]{./assets/EMichalewicz1.png} % 
		\caption{Wykres funkcji EMIchalewicz (d=5)}
		\label{fig:emichalewicz1}
	\end{center}
\end{figure}

\begin{figure}[H]
	\begin{center}
		\includegraphics[width=0.9\textwidth]{./assets/EMichalewicz2.png} % 
		\caption{Wartość znalezionego optimum w zależności od prawdopodobieństwa mutacji}
		\label{fig:emichalewicz2}
	\end{center}
\end{figure}

\begin{figure}[H]
	\begin{center}
		\includegraphics[width=0.9\textwidth]{./assets/EMichalewicz3.png} % 
		\caption{Wartość znalezionego optimum w zależności od prawdopodobieństwa krzyżowania}
		\label{fig:emichalewicz3}
	\end{center}
\end{figure}

\begin{figure}[H]
	\begin{center}
		\includegraphics[width=0.9\textwidth]{./assets/EMichalewicz4.png} % 
		\caption{Wartość znalezionego optimum w zależności od rozmiarów populacji}
		\label{fig:emichalewicz4}
	\end{center}
\end{figure}

\begin{figure}[H]
	\begin{center}
		\includegraphics[width=0.9\textwidth]{./assets/EMichalewicz5.png} % 
		\caption{Wartość znalezionego optimum w zależności od ilości iteracji}
		\label{fig:emichalewicz5}
	\end{center}
\end{figure}

\begin{figure}[H]
	\begin{center}
		\includegraphics[width=0.9\textwidth]{./assets/EMichalewicz6.png} % 
		\caption{Wartość znalezionego optimum w zależności od przyjętego elityzmu}
		\label{fig:emichalewicz6}
	\end{center}
\end{figure}

\newpage
\subsection{Hartman6 (6 parametrów)}
\paragraph{}

\begin{figure}[H]
	\begin{center}
		\includegraphics[width=0.9\textwidth]{./assets/Hartman61.png} % 
		\caption{Wykres funkcji Hartman6 (d=6)}
		\label{fig:hartman61}
	\end{center}
\end{figure}

\begin{figure}[H]
	\begin{center}
		\includegraphics[width=0.9\textwidth]{./assets/Hartman62.png} % 
		\caption{Wartość znalezionego optimum w zależności od prawdopodobieństwa mutacji}
		\label{fig:hartman62}
	\end{center}
\end{figure}

\begin{figure}[H]
	\begin{center}
		\includegraphics[width=0.9\textwidth]{./assets/Hartman63.png} % 
		\caption{Wartość znalezionego optimum w zależności od prawdopodobieństwa krzyżowania}
		\label{fig:hartman63}
	\end{center}
\end{figure}

\begin{figure}[H]
	\begin{center}
		\includegraphics[width=0.9\textwidth]{./assets/Hartman64.png} % 
		\caption{Wartość znalezionego optimum w zależności od rozmiarów populacji}
		\label{fig:hartman64}
	\end{center}
\end{figure}

\begin{figure}[H]
	\begin{center}
		\includegraphics[width=0.9\textwidth]{./assets/Hartman65.png} % 
		\caption{Wartość znalezionego optimum w zależności od ilości iteracji}
		\label{fig:hartman65}
	\end{center}
\end{figure}

\begin{figure}[H]
	\begin{center}
		\includegraphics[width=0.9\textwidth]{./assets/Hartman66.png} % 
		\caption{Wartość znalezionego optimum w zależności od przyjętego elityzmu}
		\label{fig:hartman66}
	\end{center}
\end{figure}

\newpage
\subsection{PriceTransistor (9 parametrów)}
\paragraph{}

\begin{figure}[H]
	\begin{center}
		\includegraphics[width=0.9\textwidth]{./assets/PriceTransistor1.png} % 
		\caption{Wykres funkcji PriceTransistor (d=9)}
		\label{fig:pricetransistor1}
	\end{center}
\end{figure}

\begin{figure}[H]
	\begin{center}
		\includegraphics[width=0.9\textwidth]{./assets/PriceTransistor2.png} % 
		\caption{Wartość znalezionego optimum w zależności od prawdopodobieństwa mutacji}
		\label{fig:pricetransistor2}
	\end{center}
\end{figure}

\begin{figure}[H]
	\begin{center}
		\includegraphics[width=0.9\textwidth]{./assets/PriceTransistor3.png} % 
		\caption{Wartość znalezionego optimum w zależności od prawdopodobieństwa krzyżowania}
		\label{fig:pricetransistor3}
	\end{center}
\end{figure}

\begin{figure}[H]
	\begin{center}
		\includegraphics[width=0.9\textwidth]{./assets/PriceTransistor4.png} % 
		\caption{Wartość znalezionego optimum w zależności od rozmiarów populacji}
		\label{fig:pricetransistor4}
	\end{center}
\end{figure}

\begin{figure}[H]
	\begin{center}
		\includegraphics[width=0.9\textwidth]{./assets/PriceTransistor5.png} % 
		\caption{Wartość znalezionego optimum w zależności od ilości iteracji}
		\label{fig:pricetransistor5}
	\end{center}
\end{figure}

\begin{figure}[H]
	\begin{center}
		\includegraphics[width=0.9\textwidth]{./assets/PriceTransistor6.png} % 
		\caption{Wartość znalezionego optimum w zależności od przyjętego elityzmu}
		\label{fig:pricetransistor6}
	\end{center}
\end{figure}

\newpage
\subsection{Schwefel (10 parametrów)}
\paragraph{}
Poniżej zamieszczono wzór rozpatrywanej funkcji.

\begin{equation}\label{eq:schwefel}
f(\boldsymbol{x}) = 418.9829d - \sum_{i=1}^{d} x_i \sin(\sqrt{|x_i|})
\end{equation}


\begin{figure}[H]
	\begin{center}
		\includegraphics[width=0.9\textwidth]{./assets/Schwefel1.png} % 
		\caption{Wykres funkcji Schwefel (d=10) dla dwóch pierwszych wymiarów}
		\label{fig:schwefel1}
	\end{center}
\end{figure}

\begin{figure}[H]
	\begin{center}
		\includegraphics[width=0.9\textwidth]{./assets/Schwefel2.png} % 
		\caption{Wartość znalezionego optimum w zależności od prawdopodobieństwa mutacji}
		\label{fig:schwefel2}
	\end{center}
\end{figure}

\begin{figure}[H]
	\begin{center}
		\includegraphics[width=0.9\textwidth]{./assets/Schwefel3.png} % 
		\caption{Wartość znalezionego optimum w zależności od prawdopodobieństwa krzyżowania}
		\label{fig:schwefel3}
	\end{center}
\end{figure}

\begin{figure}[H]
	\begin{center}
		\includegraphics[width=0.9\textwidth]{./assets/Schwefel4.png} % 
		\caption{Wartość znalezionego optimum w zależności od rozmiarów populacji}
		\label{fig:schwefel4}
	\end{center}
\end{figure}

\begin{figure}[H]
	\begin{center}
		\includegraphics[width=0.9\textwidth]{./assets/Schwefel5.png} % 
		\caption{Wartość znalezionego optimum w zależności od ilości iteracji}
		\label{fig:schwefel5}
	\end{center}
\end{figure}

\begin{figure}[H]
	\begin{center}
		\includegraphics[width=0.9\textwidth]{./assets/Schwefel6.png} % 
		\caption{Wartość znalezionego optimum w zależności od przyjętego elityzmu}
		\label{fig:schwefel6}
	\end{center}
\end{figure}

\newpage
\subsection{Zeldasine20 (20 parametrów)}
\paragraph{}


\begin{figure}[H]
	\begin{center}
		\includegraphics[width=0.9\textwidth]{./assets/Zeldasine201.png} % 
		\caption{Wykres funkcji Zeldasine (d=20)}
		\label{fig:zeldasine1}
	\end{center}
\end{figure}

\begin{figure}[H]
	\begin{center}
		\includegraphics[width=0.9\textwidth]{./assets/Zeldasine202.png} % 
		\caption{Wartość znalezionego optimum w zależności od prawdopodobieństwa mutacji}
		\label{fig:zeldasine2}
	\end{center}
\end{figure}

\begin{figure}[H]
	\begin{center}
		\includegraphics[width=0.9\textwidth]{./assets/Zeldasine203.png} % 
		\caption{Wartość znalezionego optimum w zależności od prawdopodobieństwa krzyżowania}
		\label{fig:zeldasine3}
	\end{center}
\end{figure}

\begin{figure}[H]
	\begin{center}
		\includegraphics[width=0.9\textwidth]{./assets/Zeldasine204.png} % 
		\caption{Wartość znalezionego optimum w zależności od rozmiarów populacji}
		\label{fig:zeldasine4}
	\end{center}
\end{figure}

\begin{figure}[H]
	\begin{center}
		\includegraphics[width=0.9\textwidth]{./assets/Zeldasine205.png} % 
		\caption{Wartość znalezionego optimum w zależności od ilości iteracji}
		\label{fig:zeldasine5}
	\end{center}
\end{figure}

\begin{figure}[H]
	\begin{center}
		\includegraphics[width=0.9\textwidth]{./assets/Zeldasine206.png} % 
		\caption{Wartość znalezionego optimum w zależności od przyjętego elityzmu}
		\label{fig:zeldasine6}
	\end{center}
\end{figure}

\newpage
\section{Podsumowanie}
\paragraph{}
Test

\fbi
Akapit

\newpage
\begin{thebibliography}{40}

\bibitem{test1}
Artur Suchwałko ,,Wprowadzenie do R dla programistów innych języków'' https://cran.r-project.org/doc/contrib/R-dla-programistow-innych-jezykow.pdf

\bibitem{test2}
Luca Scrucca ,,A quick tour of GA''
https://cran.r-project.org/web/packages/GA/vignettes/GA.html

\bibitem{test3}
Surjanovic, S. \& Bingham, D. (2013). Virtual Library of Simulation Experiments: Test Functions and Datasets. Retrieved April 3, 2017, from http://www.sfu.ca/~ssurjano.

\end{thebibliography}

\end{document}