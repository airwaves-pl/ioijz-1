\documentclass[11pt, a4paper]{article}
\usepackage{polski}
\usepackage[utf8]{inputenc}
\usepackage[T1]{fontenc}
\usepackage[export]{adjustbox}
\usepackage{graphicx}
\usepackage{amsmath} 
\usepackage{listings}
\usepackage{color}
\usepackage{marvosym}
\usepackage{geometry}
\usepackage{float}
\usepackage{booktabs}
\usepackage{multirow}
\usepackage{titlesec}
\usepackage{hyperref}
\usepackage{tabularx}

\geometry{margin=1.2in}
\usepackage[final]{pdfpages}

\newcommand{\fbi}{\leavevmode{\parindent=1em\indent}}

\definecolor{dkgreen}{rgb}{0,0.6,0}
\definecolor{gray}{rgb}{0.5,0.5,0.5}
\definecolor{mauve}{rgb}{0.58,0,0.82}

\lstset{
	frame=tblr,
	language=R,
	aboveskip=3mm,
	belowskip=3mm,
	showstringspaces=false,
	columns=flexible,
	basicstyle={\small\ttfamily},
	numbers=left,
	numberstyle=\tiny\color{gray},
	keywordstyle=\color{blue},
	commentstyle=\color{dkgreen},
	stringstyle=\color{mauve},
	breaklines=true,
	mathescape=false,
	breakatwhitespace=true,
	tabsize=3
}

\renewcommand\lstlistingname{Listing}

\titleclass{\subsubsubsection}{straight}[\subsection]
\newcounter{subsubsubsection}[subsubsection]
\renewcommand\thesubsubsubsection{\thesubsubsection.\arabic{subsubsubsection}}
\renewcommand\theparagraph{\thesubsubsubsection.\arabic{paragraph}}

\titleformat{\subsubsubsection}
  {\normalfont\normalsize\bfseries}{\thesubsubsubsection}{1em}{}
\titlespacing*{\subsubsubsection}
{0pt}{3.25ex plus 1ex minus .2ex}{1.5ex plus .2ex}

\makeatletter
\renewcommand\paragraph{\@startsection{paragraph}{5}{\z@}
  {3.25ex \@plus1ex \@minus.2ex}
  {-0em}
  {\normalfont\normalsize\bfseries}}
\renewcommand\subparagraph{\@startsection{subparagraph}{6}{\parindent}
  {3.25ex \@plus1ex \@minus .2ex}
  {-1em}
  {\normalfont\normalsize\bfseries}}
\def\toclevel@subsubsubsection{4}
\def\toclevel@paragraph{5}
\def\toclevel@paragraph{6}
\def\l@subsubsubsection{\@dottedtocline{4}{7em}{4em}}
\def\l@paragraph{\@dottedtocline{5}{10em}{5em}}
\def\l@subparagraph{\@dottedtocline{6}{14em}{6em}}
\makeatother

\setcounter{secnumdepth}{4}
\setcounter{tocdepth}{4}

\hypersetup{pageanchor=false}

\setlength\parindent{3pt}

\renewcommand{\labelenumi}{\alph{enumi}.} 

\date{\today}

\begin{document}

\begin{titlepage}

\newcommand{\HRule}{\rule{\linewidth}{0.5mm}} 
\center 

\textsc{\LARGE Politechnika Wrocławska}\\[1.5cm] 
\textsc{\Large Inteligencja Obliczeniowa i jej zastosowania}\\[0.5cm] 
\HRule \\[0.5cm]
{ \huge \bfseries Badanie algorytmu genetycznego z zakresu optymalizacji globalnej dla wybranych funkcji testowych}\\[0.5cm] 
\HRule \\[1.6cm]
 
\begin{minipage}{0.4\textwidth}
\begin{flushleft} \large
\emph{Autorzy:}\\
Paweł  \textsc{Andziul} 200648 \\
Marcin  \textsc{Słowiński} 200638 \\
\end{flushleft}
\end{minipage}
~
\begin{minipage}{0.4\textwidth}
\begin{flushright} \large
\emph{Prowadzący:} \\
dr hab. inż. Olgierd \textsc{Unold}, prof. nadzw. PWr
\end{flushright}
\end{minipage}\\[4cm]

\vfill 
{\large 12 kwietnia 2017}\\[3cm] 

\end{titlepage}

\tableofcontents

\newpage
\section{Wprowadzenie}
\paragraph{}
Algorytm genetyczny – algorytm heurystyczny, który swoim działaniem przypomina działanie ewolucji w~naturze. Osobniki będące zbyt słabe zostają wyeliminowane z~populacji w~kolejnych pokoleniach, a~na ich miejsce przyjmowane są lepsze, silniejsze, bardziej podatne adaptacji. Algorytmy te zakładają możliwość mutacji i~krzyżowania wśród potomków, przez co nie zawsze są oni silniejsi od poprzednio wyeliminowanych członków. Dodatkowo wprowadzają pojęcie elity, która jest bezpośrednio przenoszona do następnego - teoretycznie lepszego pokolenia.



\textit{dla wybranej funkcji wlasne funkcje krzyzowania (dla branina)
dla tsp (np--trudny) genetyczny -- tsplib wykorzystać do badań (2--3 instancje srednie male duze) z wlasnym operatorem z domyslnym
algorytm ga z lokalnym wyszukiwaniem, dla komiwojażera, założyć czy ma lepsze wartości, czy szybciej zbiega, jak operatory się zachowują,
psoptim, dla jednej funkcji i komiwojażera}



\fbi
W ramach laboratorium należało przeprowadzić testy algorytmu genetycznego dla różnych parametrów. Jako benchmark oceny należało użyć pakietu ,,getGlobalOpts'' oraz języka R.

\fbi
Pomiary wykonywano na 2 różnych jednostkach roboczych. Ich parametry nie są istotne z~punktu widzenia analizy i~możliwości porównania rezultatów.

\section{Implementacja}
\paragraph{}
Poniżej (listing ~\ref{lst:skryptGlowny}) zamieszczono kod napisany w~języku R przygotowany w~celu umożliwienia przeprowadzenia pomiarów.

\lstinputlisting[label=lst:skryptGlowny,caption=Skrypt w~języku R wykorzystany do badań,firstline=1,lastline=300]{./assets/skrypt_lab35.R}

\fbi
Skrypt przygotowano w~sposób który umożliwia w~pełni automatyczne przeprowadzenie wszystkich pomiarów. Jednocześnie wszystkie wykresy mogą być natychmiast podmienione w~sprawozdaniu. Poniżej pokrótce omówiono podstawowe parametry.

\begin{itemize}
	\item nOfRuns
	
	Ilość powtórzeń dla każdego pomiaru w~celu uśrednienia.
	
	\item colors, series
	
	Wektory kolorów i~nazw kolejnych serii pomiarowych. 
	
	\item params
	
	Macierz parametrów domyślnych algorytmu dla każdej z~serii. W każdym wierszu kolejno są zawarte: p. mutacji, p. krzyżowania, rozmiar populacji, ilość iteracji oraz kolor serii na wykresach.
	
	\item functions
	
	Wektor nazw funkcji dla których przeprowadzane są kolejno pomiary.
	
\end{itemize}

Całość informacji niezbędnych do przeprowadzenia obliczeń odczytywana jest na podstawie nazwy funkcji z~pakietu ,,globalOptTests''. Są to: rozmiar problemu (ilość parametrów), domyślne ograniczenia, wartość w~danym punkcie oraz optimum dla domyślnych ograniczeń.

\newpage
\section{Przebieg badań}
\paragraph{}
Do badań zostały wybrane funkcje o~różnych wymiarach zaczynając na 2 kończąc na 20. Poniżej wymieniono te funkcje wraz z~ilością wymiarów podaną w~nawiasie.

\begin{itemize}
	\item Branin (2)
	\item Gulf (3)
	\item CosMix4 (4)
	\item EMichalewicz (5)
	\item Hartman6 (6)
	\item PriceTransistor (9)
	\item Schwefel (10)
	\item Zeldasine20 (20)
\end{itemize}

\fbi
Każdy pomiar przeprowadzono 20-krotnie wyniki uśredniając co oznacza, że wartości widoczne na wykresach dla każdej serii z~osobna są uśrednione po osobnych 20 przebiegach. Domyślne parametry każdej z~serii przedstawiono poniżej (tabela~\ref{tab:parametry}). Zmianie ulegają wartości  prawdopodobieństwa mutacji i~krzyżowania by zbadać znaczenie ich obecności podczas optymalizacji.

\begin{table}[H]
	\centering
	\caption{Parametry domyślne poszczególnych serii pomiarowych}
	\label{tab:parametry}
	\begin{tabularx}{\textwidth}{|X|c|c|c|c|}
		\hline
		- & Seria 1 & Seria 2 & Seria 3 & Seria 4\\ 
		\hline
		Rozmiar populacji & 50 & 50 & 50 & 50 \\ 
		\hline 
		Rozmiar iteracji & 100 & 100 & 100 & 100 \\ 
		\hline 
		Prawdopodobieństwo mutacji & 0 & 0 & 0.1 & 0.1 \\ 
		\hline 
		Prawdopodobieństwo krzyżowania & 0 & 0.8 & 0 & 0.8 \\ 
		\hline 
	\end{tabularx} 
\end{table}

\fbi
Zielone, poziomie linie na wykresach oznaczają optima zawarte w~pakiecie ,,globalOptTests'' dla danej funkcji przy domyślnych ograniczeniach (tych samych dla których wykonywana jest optymalizacja podczas niniejszych pomiarów).

\fbi
Dla funkcji o~ilości parametrów większej niż 2 pominięto ilustracje graficzne znalezionych optimów gdyż optymalizacji podlegają wszystkie wymiary. Ilustracja dla dwóch pierwszych nie niesie ze sobą przydatnej informacji.

\newpage
\subsection{Branin (2 parametry)}
\paragraph{}
Branin jest funkcją z~dwoma parametrami. Na ilustracji (rys.~\ref{fig:branin1}) przedstawiono jej wykres a~poniżej jej wzór (\ref{eq:branin}) \cite{test4}.

\begin{equation}\label{eq:branin}
	f(\boldsymbol{x}) = a(x_2 - bx_1^2 + cx_1 - r)^2 + s(1 - t)\cos(x_1) + s
\end{equation}

, gdzie $ x_1 \in [-5, 10] $ oraz $ x_2 \in [0, 15] $.

\begin{figure}[H]
	\centering
	\includegraphics[width=0.95\textwidth]{./assets/test.png}
	\caption{Wykres funkcji Branin}
	\label{fig:branin1}
\end{figure}

\fbi
Z wykresu (rys.~\ref{fig:branin1}) wynika, że funkcja ta ma stosunkowo duży obszar w~którym może znajdować się minimum oraz dwie strefy w~których wartości są dużo większe.

\fbi
Na kolejnych stronach zamieszczono wyniki pomiarów dla różnych wartości parametrów algorytmu genetycznego. Kolejno dokonano pomiarów dla różnych wartości: prawdopodobieństwa mutacji i~krzyżowania, wielkości populacji, ilości iteracji oraz elityzmu. Wszystkie pomiary wykonano dla 4 różnych ustawień domyślnych parametrów (serie 1~--~4).

\newpage
\begin{figure}[H]
	\centering
	\includegraphics[width=0.85\textwidth]{./assets/test.png}
	\caption{Wartość znalezionego minimum funkcji Branin w~zależności od prawdopodobieństwa mutacji}
	\label{fig:branin2}
\end{figure}

\begin{figure}[H]
	\centering
	\includegraphics[width=0.85\textwidth]{./assets/test.png}
	\caption{Wartość znalezionego minimum funkcji Branin w~zależności od prawdopodobieństwa krzyżowania}
	\label{fig:branin3}
\end{figure}

\fbi
Na wykresie (rys.~\ref{fig:branin2}) można zauważyć niski wpływ ustawienia mutacji na znalezione rozwiązania. Przy wszystkich parametrach domyślnych funkcja znajduje się w~pobliżu optymalnej wartości. Miejscowe odchylenia są tu najprawdopodobniej związane z~charakterem algorytmu i~zbyt małą ilością prób poddanych uśrednieniu. Nie możemy tutaj określić czy przy wyłączonej zarówno mutacji jak i~krzyżowaniu wyniki ulegają pogorszeniu, gdyż nie ma w~tym obszarze spójności. Podobne wnioski możemy wskazać dla wykresu krzyżowania (rys.~\ref{fig:branin3}).

\begin{figure}[H]
	\centering
	\includegraphics[width=0.85\textwidth]{./assets/test.png}
	\caption{Wartość znalezionego minimum funkcji Branin w~zależności od rozmiaru populacji}
	\label{fig:branin4}
\end{figure}

\begin{figure}[H]
	\centering
	\includegraphics[width=0.85\textwidth]{./assets/test.png}
	\caption{Wartość znalezionego minimum funkcji Branin w~zależności od ilości iteracji}
	\label{fig:branin5}
\end{figure}

\fbi
Z wykresu (rys.~\ref{fig:branin4}) można odczytać podatność funkcji na zmiany rozmiaru populacji. Wyniki zbliżone do oczekiwanych zostały uzyskane dla wartości wynoszącej 45 jednostek. Widać również, że przy małej populacji znaczenie mutacji i~krzyżowania jest większe. Zauważalny jest wzrost jakości rozwiązania wraz ze wzrostem ilości jednostek populacji.

\fbi
Wykres (rys.~\ref{fig:branin5}) wskazuje wyraźną zmianę jakości rozwiązań dla 60 i~więcej iteracji. Poniżej tej wartości uzyskiwane wyniki są niestabilne, powyżej osiągają wartość zbliżoną do oczekiwanej szczególnie dla serii 4 (czyli z~włączoną mutacją i~krzyżowaniem).

\begin{figure}[H]
	\centering
	\includegraphics[width=0.85\textwidth]{./assets/test.png}
	\caption{Wartość znalezionego minimum funkcji Branin w~zależności od przyjętego elityzmu}
	\label{fig:branin6}
\end{figure}

\fbi
Z wykonanych pomiarów (rys.~\ref{fig:branin6}) wynika, że dla uzyskania optymalnego rozwiązania należy zastosować wartość elityzmu na poziomie przynajmniej 0,35. Jego ustawienie poniżej tej wartości powoduje obniżenie się jakości rezultatów.

\begin{figure}[H]
	\centering
	\includegraphics[width=0.7\textwidth]{./assets/test.png}
	\caption{Poglądowa lokalizacja najlepszego znalezionego minimum funkcji Branin dla pomiarów przy zmianach elityzmu}
	\label{fig:branin6elt}
\end{figure}

\newpage
\section{Podsumowanie}
\paragraph{}
W trakcie prowadzonych badań przetestowano algorytm genetyczny w~zadaniu optymalizacji dla 9 funkcji testowych. Analizie poddano wpływ zmiany każdego z~parametrów dla 4 różnych konfiguracji pozostałych wartości domyślnych.

\fbi
Wartość prawdopodobieństwa mutacji i~krzyżowania zdaje się odgrywać drugorzędną rolę. Istotne jednak by chociaż jedna z~nich była włączona z~prawdopodobieństwem większym niż 0.

\fbi
Najlepszym ustawieniem dla elityzmu jest prawdopodobieństwo rzędu 0,5.

\fbi
Z pewnością należałoby zwiększyć ilość prób poddawanych uśrednianiu gdyż dla przyjętych 20 wyniki ciągle są niestabilne. Warto by również rozważyć pomijanie kilku najlepszych i~najgorszych wyników przed uśrednianiem.

\fbi
Co ciekawe wyniki są widocznie gorsze przy konfiguracji w~której krzyżowanie jest wyłączone a~p. mutacji wynosi 0,5. Taka prawidłowość objawia się dla wszystkich badanych funkcji.

\newpage
\begin{thebibliography}{40}

\bibitem{test1}
Artur Suchwałko ,,Wprowadzenie do R dla programistów innych języków'' https://cran.r-project.org/doc/contrib/R-dla-programistow-innych-jezykow.pdf


\bibitem{test2}
Luca Scrucca ,,Package GA''
https://cran.r-project.org/web/packages/GA/GA.pdf

\bibitem{test3}
Surjanovic, S. \& Bingham, D. (2013). ,,Virtual Library of Simulation Experiments: Test Functions and Datasets.'' Retrieved April 3, 2017, from http://www.sfu.ca/~ssurjano.

\bibitem{test4}
Momin Jamil, Xin-She Yang ,,A literature survey of benchmark functions for global optimization problems'', Int. Journal of Mathematical Modelling and Numerical Optimisation, Vol. 4, No. 2, pp. 150–194. (2013)

\bibitem{test5}
Ajith Abraham, Aboul-Ella Hassanien, Patrick Siarry, Andries Engelbrecht, ,,Foundations of Computational Intelligence Volume 3'' (2009)

\bibitem{test6}
Onay Urfalioglu, Orhan Arikan ,,Self-adaptive randomized and rank-based differential evolution for multimodal problems'' (2011)

\end{thebibliography}

\end{document}