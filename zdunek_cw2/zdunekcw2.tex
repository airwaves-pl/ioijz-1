\documentclass[11pt, a4paper]{article}
\usepackage{polski}
\usepackage[utf8]{inputenc}
\usepackage[T1]{fontenc}
\usepackage[export]{adjustbox}
\usepackage{graphicx}
\usepackage{amsmath} 
\usepackage{listings}
\usepackage{color}
\usepackage{marvosym}
\usepackage{geometry}
\usepackage{enumerate}
\usepackage{float}
\usepackage{booktabs}
\usepackage{multirow}
\usepackage{titlesec}
\usepackage{hyperref}
\usepackage{tabularx}
\usepackage{amssymb}

\geometry{margin=1.2in}
\usepackage[final]{pdfpages}

\newcommand{\fbi}{\leavevmode{\parindent=1em\indent}}

\definecolor{dkgreen}{rgb}{0,0.6,0}
\definecolor{gray}{rgb}{0.5,0.5,0.5}
\definecolor{mauve}{rgb}{0.58,0,0.82}

\lstset{
	frame=tblr,
	language=Matlab,
	aboveskip=3mm,
	belowskip=3mm,
	showstringspaces=false,
	columns=flexible,
	basicstyle={\small\ttfamily},
	numbers=left,
	numberstyle=\tiny\color{gray},
	keywordstyle=\color{blue},
	commentstyle=\color{dkgreen},
	stringstyle=\color{mauve},
	breaklines=true,
	mathescape=false,
	breakatwhitespace=true,
	tabsize=3,
	inputencoding=utf8,
	extendedchars=true,
	literate=
	{ą}{{\k{a}}}1
	{Ą}{{\k{A}}}1
	{ę}{{\k{e}}}1
	{Ę}{{\k{E}}}1
	{ó}{{\'o}}1
	{Ó}{{\'O}}1
	{ś}{{\'s}}1
	{Ś}{{\'S}}1
	{ł}{{\l{}}}1
	{Ł}{{\L{}}}1
	{ż}{{\.z}}1
	{Ż}{{\.Z}}1
	{ź}{{\'z}}1
	{Ź}{{\'Z}}1
	{ć}{{\'c}}1
	{Ć}{{\'C}}1
	{ń}{{\'n}}1
	{Ń}{{\'N}}1
}

\renewcommand\lstlistingname{Listing}

\titleclass{\subsubsubsection}{straight}[\subsection]
\newcounter{subsubsubsection}[subsubsection]
\renewcommand\thesubsubsubsection{\thesubsubsection.\arabic{subsubsubsection}}
\renewcommand\theparagraph{\thesubsubsubsection.\arabic{paragraph}}

\titleformat{\subsubsubsection}
  {\normalfont\normalsize\bfseries}{\thesubsubsubsection}{1em}{}
\titlespacing*{\subsubsubsection}
{0pt}{3.25ex plus 1ex minus .2ex}{1.5ex plus .2ex}

\makeatletter
\renewcommand\paragraph{\@startsection{paragraph}{5}{\z@}
  {3.25ex \@plus1ex \@minus.2ex}
  {-0em}
  {\normalfont\normalsize\bfseries}}
\renewcommand\subparagraph{\@startsection{subparagraph}{6}{\parindent}
  {3.25ex \@plus1ex \@minus .2ex}
  {-1em}
  {\normalfont\normalsize\bfseries}}
\def\toclevel@subsubsubsection{4}
\def\toclevel@paragraph{5}
\def\toclevel@paragraph{6}
\def\l@subsubsubsection{\@dottedtocline{4}{7em}{4em}}
\def\l@paragraph{\@dottedtocline{5}{10em}{5em}}
\def\l@subparagraph{\@dottedtocline{6}{14em}{6em}}
\makeatother

\setcounter{secnumdepth}{4}
\setcounter{tocdepth}{4}

\hypersetup{pageanchor=false}

\setlength\parindent{3pt}

\renewcommand{\labelenumi}{\alph{enumi}.} 

\date{\today}


\begin{document}

\begin{titlepage}

\newcommand{\HRule}{\rule{\linewidth}{0.5mm}} 
\center 

\textsc{\LARGE Politechnika Wrocławska}\\[1.5cm] 
\textsc{\Large Inteligencja Obliczeniowa i jej zastosowania}\\[0.5cm] 
\HRule \\[0.5cm]
{ \huge \bfseries Ćwiczenie 2 \\*
	Metody redukcji wymiarowości \\*
	Nieujemna faktoryzacja macierzy \\i dekompozycje tensorów}\\[0.5cm] 
\HRule \\[1.6cm]
 
\begin{minipage}{0.4\textwidth}
\begin{flushleft} \large
\emph{Autorzy:}\\
Paweł  \textsc{Andziul} 200648 \\
Robert  \textsc{Chojnacki} 200685 \\
Marcin  \textsc{Słowiński} 200638 \\
\end{flushleft}
\end{minipage}
~
\begin{minipage}{0.4\textwidth}
\begin{flushright} \large
\emph{Prowadzący:} \\
dr hab. inż. Rafał \textsc{Zdunek}
\end{flushright}
\end{minipage}\\[4cm]

\vfill 
{\large 12 czerwca 2017}\\[3cm] 

\end{titlepage}

\tableofcontents

\newpage
\section{Zadanie 1}
\paragraph{}
Wygenerować faktory $A = [a_{ij}] \in R^{I x J}_{+}$ i $X = [x_{jt}] \in R^{J x T}_{+}$, gdzie $a_{ij} = max(0,\check{a}_{ij})$ i $x_{jt} = max(0,\check{x}_{jt})$ oraz $\check{a}_{ij},\check{x}_{jt}\sim N(0,1)$ (rozkład normalny). Wygeneruj syntetyczne obserwacje Y=AX dla I = 100, T = 1000, J = 10. Stosując wybrane algorytmy NMF (ALS, MUE, HALS) wyznacz estymowane faktory \^{A} i \^{X} oraz unormowany błąd residualny w funkcji iteracji naprzemiennych. Oceń jakość estymacji stosując miary MSE (ang. Mean-Squarred Error) lub SIR (ang. Signal-to-Interference Ratio).

\subsection{Algorytm ALS}
\paragraph{}

\subsection{Algorytm MUE}
\paragraph{}

\subsection{Algorytm HALS}
\paragraph{}

\subsection{Realizacja}
\paragraph{}

\subsection{Wyniki}
\paragraph{}


\section{Zadanie 2}
\paragraph{}
Wygenerować faktory..


\section{Zadanie 3}
\paragraph{}
Obrazy twarzy z bazy ORL (lub podobnej) przedstaw za pomocą tensora $Y = \in R^{I_1xI_2xI_3}$, gdzie $I_3$ jest liczbą obrazów. Rozdziel obrazy na zbiory trenujący i testujący według odpowiedniej zasady, np, 5-folds CV i utwórz odpowiednie tensory trenujący $Y_r$ i testujący $Y_t$. Tensor trenujący poddaj dekompozycji CP (np. algorytmem ALS) oraz HOSVD dla J = 4, 10, 20, 30. Pogrupować obrazy stosując metodę k-średnich dla faktora $\widehat{U}^{(3)}$. Badania przeprowadzić dla różnej liczby grup. Porównać dokładność grupowania z metodą PCA (z poprzedniego ćwiczenia). Następnie dokonaj projekcji obrazów z tensora $Y_t$ na podprzestrzeń cech generowaną faktorami otrzymanymi z $Y_r$. Dokonaj klasyfikacji obrazów w przestrzeni cech w $\widehat{U}^{(3)}$ za pomocą klasyfikatora k-NN. Porównać efekty klasyfikacji różnymi metodami (np. PCA, CP, HOSVD).

\subsection{Opis metody}
\paragraph{}

\subsection{Algorytm}
\paragraph{}

\subsection{Realizacja}
\paragraph{}

\subsection{Wyniki}
\paragraph{}

\section{Podsumowanie}
\paragraph{}
..

\newpage
\begin{thebibliography}{40}

\bibitem{test1}
\url{https://www.mathworks.com/}




\end{thebibliography}

\end{document}
